% This work is made available under the terms of the
% Creative Commons Attribution-ShareAlike 4.0 license,
% http://creativecommons.org/licenses/by-sa/4.0/.

\documentclass[a4paper]{book}

\usepackage{wrapfig}
\usepackage{graphicx}
\usepackage{hyperref}
\usepackage{multirow}
\usepackage{scalefnt}
\usepackage{tikz}

% watermark -- for draft stage
%\usepackage[firstpage]{draftwatermark}
%\SetWatermarkLightness{0.9}
%\SetWatermarkScale{5}

\input{latex_extensions}

\title{
  \textbf{ADAMS} \\
  {\Large \textbf{A}dvanced \textbf{D}ata mining \textbf{A}nd \textbf{M}achine
  learning \textbf{S}ystem} \\
  {\Large Module: adams-rest-dw} \\
  \vspace{1cm}
  \includegraphics[width=2cm]{images/rest-dw-module.png} \\
}
\author{
  Peter Reutemann
}

\setcounter{secnumdepth}{3}
\setcounter{tocdepth}{3}

\begin{document}

\begin{titlepage}
\maketitle

\thispagestyle{empty}
\center
\begin{table}[b]
	\begin{tabular}{c l l}
		\parbox[c][2cm]{2cm}{\copyright 2025} &
		\parbox[c][2cm]{5cm}{\includegraphics[width=5cm]{images/coat_of_arms.pdf}} \\
	\end{tabular}
	\includegraphics[width=12cm]{images/cc.png} \\
\end{table}

\end{titlepage}

\tableofcontents
%\listoffigures
%\listoftables

% %%%%%%%%%%%%%%%%%%%%%%%%%%%%%%%%%%
\chapter{Introduction}
REST webservices\cite{rest} are a popular variant of webservices, typically using
JSON for exchanging information.

ADAMS provides a general framework for accessing and implementing REST webservices
using Dropwizard\cite{dropwizard}.

%%%%%%%%%%%%%%%%%%%%%%%%%%%%%%%%%%%
\chapter{Using REST}
The following sections describe how you can access REST webservices (\textit{client})
and write your own ones (\textit{server}).

\section{Client}
Instead of custom Java classes for communicating with the REST server, you can use
existing flow components:

\begin{tight_itemize}
    \item \texttt{source.HTTPRequest} -- best used for \texttt{GET} requests
    with all the relevant information in the URL.
    \item \texttt{transformer.HTTPRequest} -- used for sending textual data (e.g.,
    plain text or JSON) to the server via a POST request.
\end{tight_itemize}
Both actors return a container of type \texttt{HttpRequestResult}, which typically
contains the status code/message and the response body. These can be accessed
using the \texttt{ContainerValuePicker} control actor.

Here are some useful conversion classes:
\begin{tight_itemize}
    \item \textit{MapToJson} -- converts a Java Map object into a JSON object
    \item \textit{JsonObjectToMap} -- does the reverse conversion
    \item \textit{ObjectToJson} -- serializes a Java Bean object to a JSON string
    \item \textit{JsonToObject} -- deserializes a JSON string into a Java Bean object
    \item \textit{URLEncode} -- ensures that the string can be used in a URL
    \item \textit{URLDecode} -- the reverse operation
\end{tight_itemize}

\heading{Timing}
If you want to record how long requests are taking, you can use the \textit{TimedSource}
or \textit{TimedSubProcess} actors to wrap the relevant \textit{HTTPRequest} actor.


\clearpage
\section{Server}
For defining a REST server, you need to add the \textit{DropwizardRESTServer} standalone
to your flow and then specify the relevant classes implementing the \textit{RESTProvider}
interface. The following providers are part of this module:
\begin{tight_itemize}
    \item \textit{EchoServer} -- simple example provider that just returns the data that the
    client sent via a \texttt{GET} request.
    \item \textit{GenericServer} -- To avoid having to reimplement the wheel, this provider
    takes one or more classes that implement the \textit{RESTPlugin} interface. These plugins
    will automatically get registered. In Dropwizard terms, such a plugin is referred to as
    a \textit{resource}. You can also add your own health checks
    \footnote{\url{https://www.dropwizard.io/en/stable/manual/core.html\#health-checks}{}}.
\end{tight_itemize}

\subsection{RESTPlugin}
The \textit{RESTPlugin} interface is rather light-weight, as all necessary REST information is indicated via jakarta
annotations\footnote{\url{https://jakarta.ee/learn/docs/jakartaee-tutorial/current/websvcs/rest/rest.html}{}}.
For convenience, you can simply sub-class the \textit{AbstractRESTPlugin} class (or one of its
other sub-classes). The \textit{Echo} plugin/resource for the \textit{EchoServer} looks like this:

{\scriptsize
\begin{verbatim}
@Path("/echo/{input}")  // @Path needs to be at class level!
@Produces("text/plain")
public class Echo extends AbstractRESTPlugin {

  private static final long serialVersionUID = 1L;

  public String globalInfo() {
    return "Simple echo of the input.;
  }

  @GET
  public String ping(@PathParam("input") String input) {
    return input;
  }
}
\end{verbatim}}


\subsection{Logging}
You can turn off \textit{request logging} with the following in the YAML
config file (see \cite{dropwizardconfig} for full documentation):

\begin{verbatim}
server:
  requestLog:
    appenders: []
\end{verbatim}

%%%%%%%%%%%%%%%%%%%%%%%%%%%%%%%%%%%
% Copyright (c) 2009-2012 by the University of Waikato, Hamilton, NZ. 
% This work is made available under the terms of the 
% Creative Commons Attribution-ShareAlike 3.0 license, 
% http://creativecommons.org/licenses/by-sa/3.0/. 
%
% Version: $Revision: 3353 $

\begin{thebibliography}{999}
	% to make the bibliography appear in the TOC
	\addcontentsline{toc}{chapter}{Bibliography}

    % references
	\bibitem{adams}
		\textit{ADAMS} -- Advanced Data mining and Machine learning System \\
		\url{https://adams.cms.waikato.ac.nz/}{}

	\bibitem{ffmpeg}
		\textit{FFmpeg} -- a complete, cross-platform solution to record,
		convert and stream audio and video \\
		\url{http://ffmpeg.org/}{}

	\bibitem{xuggle}
		\textit{xuggle} -- a free open-source library for Java developers
		to uncompress, manipulate, and compress recorded or live video in real time \\
		\url{http://www.xuggle.com/}{}

	\bibitem{gzip}
		\textit{gzip} -- compression/decompression algorithm based on
		the DEFLATE algorithm, which is a combination of LZ77 and Huffman
		coding. \\
		\url{https://en.wikipedia.org/wiki/Gzip}{}

\end{thebibliography}


\end{document}
