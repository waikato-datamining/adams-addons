% Copyright (c) 2023 by the University of Waikato, Hamilton, NZ.
% This work is made available under the terms of the 
% Creative Commons Attribution-ShareAlike 4.0 license,
% http://creativecommons.org/licenses/by-sa/4.0/.

\documentclass[a4paper]{book}

\usepackage{wrapfig}
\usepackage{graphicx}
\usepackage{hyperref}
\usepackage{multirow}
\usepackage{scalefnt}
\usepackage{tikz}

% watermark -- for draft stage
%\usepackage[firstpage]{draftwatermark}
%\SetWatermarkLightness{0.9}
%\SetWatermarkScale{5}

\input{latex_extensions}

\title{
  \textbf{ADAMS} \\
  {\Large \textbf{A}dvanced \textbf{D}ata mining \textbf{A}nd \textbf{M}achine
  learning \textbf{S}ystem} \\
  {\Large Module: adams-imaging-ext} \\
  \vspace{1cm}
  \includegraphics[width=2cm]{images/imaging-ext-module.png} \\
}
\author{
  Peter Reutemann
}

\setcounter{secnumdepth}{3}
\setcounter{tocdepth}{3}

\begin{document}

\begin{titlepage}
\maketitle

\thispagestyle{empty}
\center
\begin{table}[b]
	\begin{tabular}{c l l}
		\parbox[c][2cm]{2cm}{\copyright 2023} &
		\parbox[c][2cm]{5cm}{\includegraphics[width=5cm]{images/coat_of_arms.pdf}} \\
	\end{tabular}
	\includegraphics[width=12cm]{images/cc.png} \\
\end{table}

\end{titlepage}

\tableofcontents
%\listoffigures
%\listoftables

%%%%%%%%%%%%%%%%%%%%%%%%%%%%%%%%%%%
\chapter{Tools}
This module represents \textit{extended} imaging support that relies on \textit{external}
tools, like deep learning docker containers, to provide additional functionality.

%%%%%%%%%%%%%%%%%%%%%%%%%%%%%%%%%%%
\chapter(Image segmentation)
\section{DEXTR}
DEXTR\cite{dextr} (Deep Extreme Cut) is a very useful tool for aiding human annotators in
annotating images. The tool generates object masks using just four \textit{extreme points}
(x/y pairs) on the edges of the object that is to be annotated.

The \textit{ImageSegmentationAnnotator} transformer offers a DEXTR tool in its toolbox. In order
to use the tool, you need the following:

\begin{tight_itemize}
  \item A computer (preferably a Linux machine) with Redis\cite{redis} and Docker\cite{docker} running,
  this can be the local machine.
  \item Run the following flow to start a DEXTR container: \\
  \textit{adams-imaging-ext\_run-dextr.flow}.
  \item Run \textit{adams-imaging-image\_segmentation\_annotation.flow} to annotate JPEG images with the help of DEXTR.
  \begin{tight_itemize}
    \item In the annotation interface, select the \textit{DEXTR} tool.
    \item Adjust any connection parameters if necessary, though defaults work with the above flow when run on the
    local machine, and click on the tick button to connect to the Redis instance. The cursor, when hovering over
    the image, should be cross-hair now.
    \item With the left mouse button, perform four left-clicks to choose the four extreme points (use the right mouse
    button to remove the current points and start over) and hit \textit{ENTER} to send image and points to the docker
    container for processing (on a CPU, this can take a couple of seconds).
    \item If successful, the mask will be added to the current layer.
  \end{tight_itemize}
\end{tight_itemize}

%%%%%%%%%%%%%%%%%%%%%%%%%%%%%%%%%%%
\chapter(Object detection)
\section{OPEX}
The \textit{OPEX} tool, available through the interface of the \textit{ImageObjectAnnotator} transformer,
 allows you to send the current image to a docker container that runs a deep learning model which returns
 object detection predictions in OPEX format\cite{opex}.

In order to use the tool, you need the following:

\begin{tight_itemize}
  \item A computer (preferably a Linux machine) with Redis\cite{redis} and Docker\cite{docker} running,
  this can be the local machine.
  \item Run the following flow to start an OPEX-outputting container, e.g., using Yolov5: \\
  \textit{adams-imaging-ext\_run-yolov5.flow}.
  \item Run \textit{adams-imaging-annotate\_objects.flow} to annotate JPEG images with the help of OPEX.
  \begin{tight_itemize}
    \item In the annotation interface, select the \textit{OPEX} tool.
    \item Adjust any connection parameters if necessary, though defaults work with the above flow when run on the
    local machine, and click on the tick button to connect to the Redis instance. The cursor, when hovering over
    the image, should be cross-hair now.
    \item Click anywhere on the image while holding the \textit{SHIFT} key down, to send the image to the docker
    container for processing (even on a CPU, this will be very quick).
    \item If successful, the predicted bounding boxes will be overlayed over the image.
  \end{tight_itemize}
\end{tight_itemize}


\section{DEXTR}
DEXTR\cite{dextr} (Deep Extreme Cut) is a very useful tool for aiding human annotators in
annotating images. The tool generates object masks using just four \textit{extreme points}
(x/y pairs) on the edges of the object that is to be annotated. These masks can be converted into polygons
and integrated into the annotations.

The \textit{ImageObjectAnnotator} transformer offers a DEXTR tool in its toolbox. In order
to use the tool, you need the following:

\begin{tight_itemize}
  \item A computer (preferably a Linux machine) with Redis\cite{redis} and Docker\cite{docker} running,
  this can be the local machine.
  \item Run the following flow to start a DEXTR container: \\
  \textit{adams-imaging-ext\_run-dextr.flow}.
  \item Run \textit{adams-imaging-annotate\_objects.flow} to annotate JPEG images with the help of DEXTR.
  \begin{tight_itemize}
    \item In the annotation interface, select the \textit{DEXTR} tool.
    \item Adjust any connection parameters if necessary, though defaults work with the above flow when run on the
    local machine, and click on the tick button to connect to the Redis instance. The cursor, when hovering over
    the image, should be cross-hair now.
    \item With the left mouse button, perform four left-clicks to choose the four extreme points (use the left mouse
    button while holding CTRL to remove the current points and start over) and hit \textit{ENTER} to send image and
    points to the docker container for processing (on a CPU, depending on the image size, this can take a couple of seconds).
    \item If successful, the annotation(s) will be added to the current ones.
  \end{tight_itemize}
\end{tight_itemize}


%%%%%%%%%%%%%%%%%%%%%%%%%%%%%%%%%%%
% Copyright (c) 2009-2012 by the University of Waikato, Hamilton, NZ. 
% This work is made available under the terms of the 
% Creative Commons Attribution-ShareAlike 3.0 license, 
% http://creativecommons.org/licenses/by-sa/3.0/. 
%
% Version: $Revision: 3353 $

\begin{thebibliography}{999}
	% to make the bibliography appear in the TOC
	\addcontentsline{toc}{chapter}{Bibliography}

    % references
	\bibitem{adams}
		\textit{ADAMS} -- Advanced Data mining and Machine learning System \\
		\url{https://adams.cms.waikato.ac.nz/}{}

	\bibitem{ffmpeg}
		\textit{FFmpeg} -- a complete, cross-platform solution to record,
		convert and stream audio and video \\
		\url{http://ffmpeg.org/}{}

	\bibitem{xuggle}
		\textit{xuggle} -- a free open-source library for Java developers
		to uncompress, manipulate, and compress recorded or live video in real time \\
		\url{http://www.xuggle.com/}{}

	\bibitem{gzip}
		\textit{gzip} -- compression/decompression algorithm based on
		the DEFLATE algorithm, which is a combination of LZ77 and Huffman
		coding. \\
		\url{https://en.wikipedia.org/wiki/Gzip}{}

\end{thebibliography}


\end{document}
