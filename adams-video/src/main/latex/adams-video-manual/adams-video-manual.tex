% Copyright (c) 2015 by the University of Waikato, Hamilton, NZ.
% This work is made available under the terms of the 
% Creative Commons Attribution-ShareAlike 3.0 license, 
% http://creativecommons.org/licenses/by-sa/3.0/. 
%
% Version: $Revision: 3363 $

\documentclass[a4paper]{book}

\usepackage{wrapfig}
\usepackage{graphicx}
\usepackage{hyperref}
\usepackage{multirow}
\usepackage{scalefnt}
\usepackage{tikz}
\usepackage{varwidth}

% watermark -- for draft stage
\usepackage[firstpage]{draftwatermark}
\SetWatermarkLightness{0.9}
\SetWatermarkScale{5}

\input{latex_extensions}

\title{
  \textbf{ADAMS} \\
  {\Large \textbf{A}dvanced \textbf{D}ata mining \textbf{A}nd \textbf{M}achine
  learning \textbf{S}ystem} \\
  {\Large Module: adams-video} \\
  \vspace{1cm}
  \includegraphics[width=2cm]{images/video-module.png} \\
}
\author{
  Peter Reutemann
}

\setcounter{secnumdepth}{3}
\setcounter{tocdepth}{3}

\begin{document}

\begin{titlepage}
\maketitle

\thispagestyle{empty}
\center
\begin{table}[b]
	\begin{tabular}{c l l}
		\parbox[c][2cm]{2cm}{\copyright 2015} &
		\parbox[c][2cm]{5cm}{\includegraphics[width=5cm]{images/coat_of_arms.pdf}} \\
	\end{tabular}
	\includegraphics[width=12cm]{images/cc.png} \\
\end{table}

\end{titlepage}

\tableofcontents
%\listoffigures
%\listoftables

%%%%%%%%%%%%%%%%%%%%%%%%%%%%%%%%%%%
\chapter{Flow}
The video module offers some actors for basic video display and processing support.

\noindent Available sources:
\begin{tight_itemize}
    \item \texttt{ListWebcams} -- lists the names of all available webcams
    attached to the computer\footnote{adams-video-list\_webcams.flow}.
    \item \texttt{WebcamImage} -- outputs images from the selected webcam
    attached to the computer\footnote{adams-video-webcam.flow}.
    \item \texttt{WebcamInfo} -- outputs images from the selected webcam
    \footnote{adams-video-webcam\_info.flow}.
\end{tight_itemize}

\noindent Available transformers:
\begin{tight_itemize}
	\item \texttt{AddTrailBackground} -- adds a image as trail background
	\item \texttt{AddTrailStep} -- adds an additional step to the trail
	passing through.
	\item \texttt{ExtractTrackedObject} -- extracts a tracked
	object in an image and forwards it as new image
	container\footnote{adams-video-track\_objects-user\_selected\_object.flow}.
	\item \texttt{GetTrailBackground} -- retrieves the trail background image, if any.
	\item \texttt{MjpegImageSequence} -- generates an image sequence
	from MJPEG movies, one frame at a time\footnote{adams-video-play\_mjpeg\_video.flow}.
	\item \texttt{MovieImageSequence} -- generates an image sequence
	from movies, one frame at a time (uses Xuggle\cite{xuggle}\footnote{adams-video-play\_mp4\_video.flow}).
	\item \texttt{TrackObjects} -- tracks objects in images sequences,
	e.g., from movies\footnote{adams-video-track\_objects-predefined.flow, adams-video-track\_objects-predefined2.flow}.
	\item \texttt{TrailFileReader} -- reads a trail from disk.
	\item \texttt{TrailFileWriter} -- writes a trail to disk.
	\item \texttt{TrailFilter} -- applies a filter to the trail passing through.
	\item \texttt{TransformTrackedObject} -- transforms a tracked
	object in an image with a callable transformer, e.g., for blurring a
	face\footnote{adams-video-track\_objects-predefined2.flow}.
\end{tight_itemize}

\noindent Available sinks:
\begin{tight_itemize}
  \item \textit{FFmpeg} -- actor for processing videos using
  ffmpeg\cite{ffmpeg}\footnote{adams-video-ffmpeg.flow}.
\end{tight_itemize}

\noindent Available conversions:
\begin{tight_itemize}
  \item \textit{QuadrilateralLocationCenter} -- outputs a Point2D object that
  is the center of the rectangle surrounding the quadrilateral coordinates.
  \item \textit{QuadrilateralLocationToString} -- turns the quadrilateral
  coordinates into a string.
  \item \textit{StringToQuadrilateralLocation} -- turns a string into quadrilateral
  locations.
\end{tight_itemize}


%%%%%%%%%%%%%%%%%%%%%%%%%%%%%%%%%%%
% Copyright (c) 2009-2012 by the University of Waikato, Hamilton, NZ. 
% This work is made available under the terms of the 
% Creative Commons Attribution-ShareAlike 3.0 license, 
% http://creativecommons.org/licenses/by-sa/3.0/. 
%
% Version: $Revision: 3353 $

\begin{thebibliography}{999}
	% to make the bibliography appear in the TOC
	\addcontentsline{toc}{chapter}{Bibliography}

    % references
	\bibitem{adams}
		\textit{ADAMS} -- Advanced Data mining and Machine learning System \\
		\url{https://adams.cms.waikato.ac.nz/}{}

	\bibitem{ffmpeg}
		\textit{FFmpeg} -- a complete, cross-platform solution to record,
		convert and stream audio and video \\
		\url{http://ffmpeg.org/}{}

	\bibitem{xuggle}
		\textit{xuggle} -- a free open-source library for Java developers
		to uncompress, manipulate, and compress recorded or live video in real time \\
		\url{http://www.xuggle.com/}{}

	\bibitem{gzip}
		\textit{gzip} -- compression/decompression algorithm based on
		the DEFLATE algorithm, which is a combination of LZ77 and Huffman
		coding. \\
		\url{https://en.wikipedia.org/wiki/Gzip}{}

\end{thebibliography}


\end{document}
