% This work is made available under the terms of the
% Creative Commons Attribution-ShareAlike 4.0 license,
% http://creativecommons.org/licenses/by-sa/4.0/.

\documentclass[a4paper]{book}

\usepackage{wrapfig}
\usepackage{graphicx}
\usepackage{hyperref}
\usepackage{multirow}
\usepackage{scalefnt}
\usepackage{tikz}

% watermark -- for draft stage
%\usepackage[firstpage]{draftwatermark}
%\SetWatermarkLightness{0.9}
%\SetWatermarkScale{5}

\input{latex_extensions}

\title{
  \textbf{ADAMS} \\
  {\Large \textbf{A}dvanced \textbf{D}ata mining \textbf{A}nd \textbf{M}achine
  learning \textbf{S}ystem} \\
  {\Large Module: adams-rabbitmq} \\
  \vspace{1cm}
  \includegraphics[width=2cm]{images/rabbitmq-module.png} \\
}
\author{
  Peter Reutemann
}

\setcounter{secnumdepth}{3}
\setcounter{tocdepth}{3}

\begin{document}

\begin{titlepage}
\maketitle

\thispagestyle{empty}
\center
\begin{table}[b]
	\begin{tabular}{c l l}
		\parbox[c][2cm]{2cm}{\copyright 2019} &
		\parbox[c][2cm]{5cm}{\includegraphics[width=5cm]{images/coat_of_arms.pdf}} \\
	\end{tabular}
	\includegraphics[width=12cm]{images/cc.png} \\
\end{table}

\end{titlepage}

\tableofcontents
%\listoffigures
%\listoftables

%%%%%%%%%%%%%%%%%%%%%%%%%%%%%%%%%%%
\chapter{Introduction}
RabbitMQ\cite{rabbitmq} is an open source message broker. It supports:
\begin{tight_itemize}
  \item multiple messaging protocols
  \item message queuing
  \item delivery acknowledgement
  \item flexible routing to queues
  \item multiple exchange type
\end{tight_itemize}
ADAMS makes use of it for publishing and consuming messages, as well as for
remote procedure calls (RPC).

%%%%%%%%%%%%%%%%%%%%%%%%%%%%%%%%%%%
\chapter{Installation}
Installation instructions for all available platforms can be found here: \\
\url{https://www.rabbitmq.com/download.html}{}

\section{Linux}
You can install the RabbitMQ server on Debian systems like this:
\begin{verbatim}
sudo apt-get install rabbitmq-server
\end{verbatim}

\noindent For RPM-based distributions, please follow the instructions here: \\
\url{https://www.rabbitmq.com/install-rpm.html}{}

\section{Windows}
For installing the server on Windows, please follow the instructions here: \\
\url{https://www.rabbitmq.com/install-windows.html}{}

\section{TLS Support}
\label{tls_support}
You can find information on how to set up RabbitMQ using TLS on the
following URL: \\
\url{https://www.rabbitmq.com/ssl.html}{}

%%%%%%%%%%%%%%%%%%%%%%%%%%%%%%%%%%%
\chapter{Flow}
The following standalones are available:
\begin{verbatim}
  \item \textit{RabbitMQConnection} -- defines a RabbitMQ broker connection.
  \item \textit{RabbitMQChannelAction} -- peforms the specified channel action.
\end{verbatim}

The following sources are available:
\begin{verbatim}
  \item \textit{RabbitMQConsume} -- outputs the data it consumes from either
  a queue or an exchange.
\end{verbatim}

The following transformers are available:
\begin{verbatim}
  \item \textit{RabbitMQRemoteProcedureCall} -- for performing remote procedure
  calls via a RabbitMQ broker.
\end{verbatim}

The following sinks are available:
\begin{verbatim}
  \item \textit{RabbitMQMessageDeliveryAction} -- for executing an action using
  the incoming delivery tag, e.g., acknowledging it.
  \item \textit{RabbitMQPublish} -- publishes the incoming data to a queue
  or an exchange.
\end{verbatim}

The following control actors are available:
\begin{verbatim}
  \item \textit{RabbitMQRemoteSubProcess} -- executes the sub-actors remotely
  for processing the input, forwards the generated output.
  \item \textit{RabbitMQRemoteTee} -- executes the sub-actors remotely
  for processing the input, forwards the input.
  \item \textit{RabbitMQRemoteTrigger} -- executes the sub-actors remotely,
  forwards the input.
\end{verbatim}

The following conversions are available:
\begin{verbatim}
  \item \textit{RabbitMQEnvelopeToMap} -- converts RabbitMQ envelope into a
  Java Map, e.g., to obtain a delivery tag to acknowledge consumption of a message.
  \item \textit{RabbitMQPropertiesToMap} -- converts RabbitMQ basic properties
  into a Java Map, e.g., to obtain the reply-to queue name.
\end{verbatim}

The following JobRunners are available:
\begin{verbatim}
  \item \textit{RabbitMQJobRunner} -- distributes the jobs using the specified
  broker.
\end{verbatim}

\section{TLS Support}
If you want to take advantage of TLS, i.e., encrypting your connections to
the RabbitMQ server, then you need to configure your server for that (see
\ref{tls_support}) and encapsulate your actual connection factory used by
\textit{RabbitMQConnection} in one of the following meta-factories:
\begin{tight_itemize}
  \item \textit{NonValidatingSSLConnectionFactory} -- only to be used for testing/development
  \item \textit{SSLConnectionFactory}
\end{tight_itemize}

\section{Large payloads}
In theory, you can use RabbitMQ to send any amount of data. However, queues
may be constrained in disk size and sending large amounts (e.g., processing
large datasets) will also slow down the overall performance of the message broker.

As long as the sending and receiving ends have access to the same shared directory,
it is possible to exchange the actual payload via this directory and only send
the filename via the message broker. For this scenario, you can use the following
converters:
\begin{tight_itemize}
  \item adams.core.net.rabbitmq.send.FileBasedConverter
  \item adams.core.net.rabbitmq.receive.FileBasedConverter
\end{tight_itemize}
The receiving converter automatically deletes the payload file ones read
successfully.

%%%%%%%%%%%%%%%%%%%%%%%%%%%%%%%%%%%
% Copyright (c) 2009-2012 by the University of Waikato, Hamilton, NZ. 
% This work is made available under the terms of the 
% Creative Commons Attribution-ShareAlike 3.0 license, 
% http://creativecommons.org/licenses/by-sa/3.0/. 
%
% Version: $Revision: 3353 $

\begin{thebibliography}{999}
	% to make the bibliography appear in the TOC
	\addcontentsline{toc}{chapter}{Bibliography}

    % references
	\bibitem{adams}
		\textit{ADAMS} -- Advanced Data mining and Machine learning System \\
		\url{https://adams.cms.waikato.ac.nz/}{}

	\bibitem{ffmpeg}
		\textit{FFmpeg} -- a complete, cross-platform solution to record,
		convert and stream audio and video \\
		\url{http://ffmpeg.org/}{}

	\bibitem{xuggle}
		\textit{xuggle} -- a free open-source library for Java developers
		to uncompress, manipulate, and compress recorded or live video in real time \\
		\url{http://www.xuggle.com/}{}

	\bibitem{gzip}
		\textit{gzip} -- compression/decompression algorithm based on
		the DEFLATE algorithm, which is a combination of LZ77 and Huffman
		coding. \\
		\url{https://en.wikipedia.org/wiki/Gzip}{}

\end{thebibliography}


\end{document}
