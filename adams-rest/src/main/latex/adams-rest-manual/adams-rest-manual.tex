% This work is made available under the terms of the
% Creative Commons Attribution-ShareAlike 4.0 license,
% http://creativecommons.org/licenses/by-sa/4.0/.

\documentclass[a4paper]{book}

\usepackage{wrapfig}
\usepackage{graphicx}
\usepackage{hyperref}
\usepackage{multirow}
\usepackage{scalefnt}
\usepackage{tikz}

% watermark -- for draft stage
%\usepackage[firstpage]{draftwatermark}
%\SetWatermarkLightness{0.9}
%\SetWatermarkScale{5}

\input{latex_extensions}

\title{
  \textbf{ADAMS} \\
  {\Large \textbf{A}dvanced \textbf{D}ata mining \textbf{A}nd \textbf{M}achine
  learning \textbf{S}ystem} \\
  {\Large Module: adams-rest} \\
  \vspace{1cm}
  \includegraphics[width=2cm]{images/rest-module.png} \\
}
\author{
  Peter Reutemann
}

\setcounter{secnumdepth}{3}
\setcounter{tocdepth}{3}

\begin{document}

\begin{titlepage}
\maketitle

\thispagestyle{empty}
\center
\begin{table}[b]
	\begin{tabular}{c l l}
		\parbox[c][2cm]{2cm}{\copyright 2018-2019} &
		\parbox[c][2cm]{5cm}{\includegraphics[width=5cm]{images/coat_of_arms.pdf}} \\
	\end{tabular}
	\includegraphics[width=12cm]{images/cc.png} \\
\end{table}

\end{titlepage}

\tableofcontents
%\listoffigures
%\listoftables

% %%%%%%%%%%%%%%%%%%%%%%%%%%%%%%%%%%
\chapter{Introduction}
REST webservices (\cite{rest}) are a popular variant of webservices, that
are quite often easier to implement than full-blown SOAP-based ones (\cite{soap}).

ADAMS provides a general framework for accessing and implementing REST webservices
using Apache CXF\cite{cxf}.

%%%%%%%%%%%%%%%%%%%%%%%%%%%%%%%%%%%
\chapter{Using REST}
The following sections describe how you can access REST webservices (\textit{client})
and write your own ones (\textit{server}).

\section{Client}
There are two options for accessing a webservice: custom code for sending/receiving
data or via generic processing in the flow itself.

\subsection{Custom code}
When using custom code, you can use one of the following superclasses to derive
your own code from:
\begin{tight_itemize}
  \item adams.flow.rest.AbstractRESTClientSource
  \item adams.flow.rest.AbstractRESTClientTransformer
  \item adams.flow.rest.AbstractRESTClientSink
\end{tight_itemize}
These classes are generics and require you to supply the input and/or output
types, override methods for integrating them in the flow (accepts/generates)
and methods for handling input/output data.
These superclasses are available in the flow through the following, corresponding
actors:
\begin{tight_itemize}
  \item adams.flow.source.RESTSource
  \item adams.flow.transformer.RESTTransformer
  \item adams.flow.sink.RESTSink
\end{tight_itemize}

The \textit{adams.flow.rest.echo.EchoClientTransformer} class is a simple
example that sends a UTF-8 string it receives to an Echo REST server, which simply
returns the same data. In the \verb|doQuery| method, a URL is constructed
from the URL of the echo server and the URL-encoded string that is to be sent
to the server. The actual sending via the \verb|GET| method is handled by the
\verb|adams.core.net.HttpRequestHelper| class. The response from the echo
server is then decoded from a UTF-8 string and forwarded in the flow.

\clearpage
{\scriptsize
\begin{verbatim}
package adams.flow.rest.echo;
import adams.core.base.BaseURL;
import adams.core.net.HttpRequestHelper;
import adams.flow.container.HTMLRequestResult;
import adams.flow.rest.AbstractRESTClientTransformer;
import org.jsoup.Connection.Method;
import java.net.URLDecoder;
import java.net.URLEncoder;

public class EchoClientTransformer
  extends AbstractRESTClientTransformer<String,String> {

  protected String m_RequestData;

  public String globalInfo() { return "Client (transformer) for Echo REST service."; }

  public Class[] accepts() { return new Class[]{String.class}; }

  public Class[] generates() { return new Class[]{String.class}; }

  public void setRequestData(String value) { m_RequestData = value; }

  protected void doQuery() throws Exception {
    String url;
    if (getUseAlternativeURL())
      url = getAlternativeURL();
    else
      url = new EchoServer().getDefaultURL();
    url += "echo/" + URLEncoder.encode(m_RequestData, "UTF-8");
    HTMLRequestResult result = HttpRequestHelper.send(new BaseURL(url), Method.GET, null, null);
    if (result.getValue(HTMLRequestResult.VALUE_STATUSCODE, Integer.class) == 200)
      setResponseData(URLDecoder.decode(result.getValue(HTMLRequestResult.VALUE_BODY, String.class), "UTF-8"));
    else
      m_LastError = result.getValue(HTMLRequestResult.VALUE_STATUSCODE) + ": "
          + result.getValue(HTMLRequestResult.VALUE_STATUSMESSAGE);
  }
}
\end{verbatim}}

\subsection{Generic flow}
Using the above example of accessing an echo server, we can simply use existing
component available through the flow\footnote{See example flow: adams-rest-use\_service.flow}:
\begin{tight_itemize}
  \item The string to be sent to the server would be encoded via the \verb|URLEncode|
  conversion.
  \item This string would then be prefixed with the URL of the actual server
  (e.g., \verb|http://localhost:8080/echo/|) to construct the complete URL.
  \item Via the \verb|HTTPRequest| source, you can then connect to the
  complete URL. Choose the correct method for accessing, like \verb|GET|
  or \verb|POST|. This source actor allows you to attach additional headers and
  parameters as key-value pairs to your request. Cookies are accessed through
  storage, expecting a map object.
  \item Using the \verb|ContainerValuePicker| control actor, the response
  returned by the echo server can be retrieved.
  \item With the \verb|URLDecode| conversion, the URL encoded response can
  be turned into a regular string again.
\end{tight_itemize}


\clearpage
\section{Server}
For implementing the server side, you can use two different approaches:
\begin{tight_itemize}
  \item Custom class to encapsulate complete server
  \item Creating fine-grained plugins for the \verb|GenericServer|
\end{tight_itemize}
For both approaches, it is recommended to look at the some of the tutorials
listed on the Wikipedia article on JAX-RS\cite{jax-rs}, the Java API for
RESTful Web Services, to get an understanding on how to code for REST.

JAX-RS handles everything, methods (GET/POST), paths and path parameters through
Java annotations, which you will see in the following examples.

\subsection{Custom class}
In order to make the custom REST service available through the \verb|RESTServer|
standalone, it needs to be either sub-classed from \verb|AbstractRESTProvider|
or implement \verb|RESTProvider| interface (both are located in
\verb|adams.flow.rest|). In the following the use of \verb|AbstractRESTProvider|
is discussed, as it simplifies the implementation and supports handling for
custom URLs, interceptors, etc.

The \verb|doStart| method configures an \verb|JAXRSServerFactoryBean| (located
in package \verb|org.apache.cxf.jaxrs|) instance and returns the
\verb|Server| (package \verb|org.apache.cxf.endpoint|) generated from it.
The functionality to be offered by the REST service is defined by the
service bean that is set via the \verb|setServiceBean| method or the beans
that are set via the \verb|setServiceBeans| method. Optional TLS support,
i.e., serving content via \textit{https} is configured by calling the \verb|configureTLS|
method with the configured factory bean (see \ref{tls_support} for more details).

{\scriptsize
\begin{verbatim}
package adams.flow.rest;

import adams.core.Utils;
import adams.flow.rest.AbstractRESTProvider;
import org.apache.cxf.endpoint.Server;
import org.apache.cxf.jaxrs.JAXRSServerFactoryBean;
import somewhere.Echo;

public class EchoServer extends AbstractRESTProvider {

  public String globalInfo() {
    return "Simple echo server.";
  }

  public String getDefaultURL() {
    return "http://localhost:8080/";
  }

  protected Server doStart() throws Exception {
    JAXRSServerFactoryBean factory = new JAXRSServerFactoryBean();
    configureInterceptors(factory);
    factory.setServiceBean(new Echo());
    factory.setAddress(getURL());
    configureTLS(factory);
    return factory.create();
  }
}
\end{verbatim}}

\clearpage
{\scriptsize
\begin{verbatim}
package somewhere;

public class Echo {

  @GET
  @Path("/echo/{input}")
  @Produces("text/plain")
  public String ping(@PathParam("input") String input) {
    return input;
  }
}
\end{verbatim}}

\subsection{Generic server}
The light-weight approach of REST and being able to just supply an arbitrary
number of service beans, makes it possible to break up monolithic servers
into micro-servers and opening up the possibility of given the user the choice
in the flow over what functionality the server shoudl have. For that purpose,
the \verb|GenericServer| (package \verb|adams.flow.rest|) was developed.
In order to add functionality to this REST server, plugins need to be developed.
These plugins either just implement the \verb|RESTPlugin| interface or are
sub-classed from \verb|AbstractRESTPlugin| (also in package \verb|adams.flow.rest|).

The following code shows the plugin for out echo server, which would be selected
as plugin in our \verb|GenericServer|, which in turn is configured through the
\verb|RESTServer| standalone.

{\scriptsize
\begin{verbatim}
package adams.flow.rest;

import adams.flow.rest.AbstractRESTPlugin;

import javax.ws.rs.GET;
import javax.ws.rs.Path;
import javax.ws.rs.PathParam;
import javax.ws.rs.Produces;

public class Echo extends AbstractRESTPlugin {

  public String globalInfo() {
    return "Simple echo of the input.";
  }

  @GET
  @Path("/echo/{input}")
  @Produces("text/plain")
  public String ping(@PathParam("input") String input) {
    getLogger().info("input: " + input);
    return input;
  }
}
\end{verbatim}}

\noindent There are more abstract superclasses available for plugins, to
avoid unnecessary duplication:
\begin{tight_itemize}
  \item \textit{AbstractRegisteredFlowRESTPlugin} -- allows retrieval of
  registered flows via their ID.
  \item \textit{AbstractRESTPluginWithDatabaseConnection} -- enables database
  access via its flow context and the available \textit{DatabaseConnection}
  standalone.
  \item \textit{AbstractRESTPluginWithFlowContext} -- for any plugin that
  requires flow context, e.g., for accessing configurations.
\end{tight_itemize}

\subsection{Context}
If your REST service components require a context, e.g., flow context, then
you have to use the following approach for setting up the service
factory\footnote{\url{http://cxf.apache.org/docs/jaxrs-services-configuration.html}{}}:
\begin{verbatim}
JAXRSServerFactoryBean sf = new JAXRSServerFactoryBean();
CustomerService cs = new CustomerService();
// HERE: set context in CustomerService service bean
sf.setServiceBean(cs);
sf.setAddress("http://localhost:9080/");
sf.create();
\end{verbatim}
By instantiating the \textit{beans} yourself rather than through the factory,
you can provide them with context. You then use the \textit{setServiceBean(s)}
methods to set one or more beans.

%%%%%%%%%%%%%%%%%%%%%%%%%%%%%%%%%%%
\chapter{Flow}
This module contains generic actors in which you can simply plug your 
web-services that you have implemented. In the following a short overview.

The following standalones are available:
\begin{tight_itemize}
	\item \textit{RESTServer} -- runs a web-service. waiting for
	requests\footnote{adams-rest-server.flow}.
\end{tight_itemize}
The following sources are available:
\begin{tight_itemize}
	\item \textit{RESTSource} -- queries a web-service and forwards the received
	data\footnote{adams-rest-echo\_source.flow}.
\end{tight_itemize}
The following transformers are available:
\begin{tight_itemize}
	\item \textit{RESTTransformer} -- sends the data it receives to a web-service
	and forwards the data from the response in
	turn\footnote{adams-rest-echo\_transformer.flow}.
\end{tight_itemize}
The following sinks are available:
\begin{tight_itemize}
	\item \textit{RESTSink} -- simply sends data to a web-service\footnote{adams-rest-echo\_sink.flow}.
\end{tight_itemize}

\section{TLS support}
\label{tls_support}
TLS support is automatically configured in case the URL uses \verb|https://| as
protocol and the \verb|RESTUtils.configureClient| method is called. For the
server side, you need to call the \verb|AbstractRESTProvider.configureTLS| method
after the URL has been set for the \verb|JAXRSServerFactoryBean| factory instance.
The following standalones have to be present (and configured) to successfully
set up TLS support:
\begin{tight_itemize}
  \item \textit{KeyManager}
  \item \textit{TrustManager}
  \item \textit{SSLContext} -- optional, only used to determine the version of the
  TLS protocol (fallback is \textit{TSL}).
\end{tight_itemize}
See the following flows for example setups using self-signed certificates:
\begin{tight_itemize}
  \item \textit{adams-rest-generic\_server-ssl.flow} -- echo server via https
  \item \textit{adams-rest-use\_service-ssl.flow} -- access the echo server
\end{tight_itemize}
The \verb|README.md| file in the \verb|${FLOWS}/restssl| directory contains
more information on how these self-signed certificates were generated.


%%%%%%%%%%%%%%%%%%%%%%%%%%%%%%%%%%%
% Copyright (c) 2009-2012 by the University of Waikato, Hamilton, NZ. 
% This work is made available under the terms of the 
% Creative Commons Attribution-ShareAlike 3.0 license, 
% http://creativecommons.org/licenses/by-sa/3.0/. 
%
% Version: $Revision: 3353 $

\begin{thebibliography}{999}
	% to make the bibliography appear in the TOC
	\addcontentsline{toc}{chapter}{Bibliography}

    % references
	\bibitem{adams}
		\textit{ADAMS} -- Advanced Data mining and Machine learning System \\
		\url{https://adams.cms.waikato.ac.nz/}{}

	\bibitem{ffmpeg}
		\textit{FFmpeg} -- a complete, cross-platform solution to record,
		convert and stream audio and video \\
		\url{http://ffmpeg.org/}{}

	\bibitem{xuggle}
		\textit{xuggle} -- a free open-source library for Java developers
		to uncompress, manipulate, and compress recorded or live video in real time \\
		\url{http://www.xuggle.com/}{}

	\bibitem{gzip}
		\textit{gzip} -- compression/decompression algorithm based on
		the DEFLATE algorithm, which is a combination of LZ77 and Huffman
		coding. \\
		\url{https://en.wikipedia.org/wiki/Gzip}{}

\end{thebibliography}


\end{document}
