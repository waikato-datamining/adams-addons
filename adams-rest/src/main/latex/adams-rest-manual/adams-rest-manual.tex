% This work is made available under the terms of the
% Creative Commons Attribution-ShareAlike 4.0 license,
% http://creativecommons.org/licenses/by-sa/4.0/.

\documentclass[a4paper]{book}

\usepackage{wrapfig}
\usepackage{graphicx}
\usepackage{hyperref}
\usepackage{multirow}
\usepackage{scalefnt}
\usepackage{tikz}

% watermark -- for draft stage
\usepackage[firstpage]{draftwatermark}
\SetWatermarkLightness{0.9}
\SetWatermarkScale{5}

\input{latex_extensions}

\title{
  \textbf{ADAMS} \\
  {\Large \textbf{A}dvanced \textbf{D}ata mining \textbf{A}nd \textbf{M}achine
  learning \textbf{S}ystem} \\
  {\Large Module: adams-rest} \\
  \vspace{1cm}
  \includegraphics[width=2cm]{images/rest-module.png} \\
}
\author{
  Peter Reutemann
}

\setcounter{secnumdepth}{3}
\setcounter{tocdepth}{3}

\begin{document}

\begin{titlepage}
\maketitle

\thispagestyle{empty}
\center
\begin{table}[b]
	\begin{tabular}{c l l}
		\parbox[c][2cm]{2cm}{\copyright -2017} &
		\parbox[c][2cm]{5cm}{\includegraphics[width=5cm]{images/coat_of_arms.pdf}} \\
	\end{tabular}
	\includegraphics[width=12cm]{images/cc.png} \\
\end{table}

\end{titlepage}

\tableofcontents
\listoffigures
%\listoftables

% %%%%%%%%%%%%%%%%%%%%%%%%%%%%%%%%%%
\chapter{Introduction}
REST webservices (\cite(rest)) are a popular variant of webservices, that
are quite often easier to implement than full-blown SOAP-based ones (\cite(soap)).

ADAMS provides a general framework for accessing and implementing REST webservices.

%%%%%%%%%%%%%%%%%%%%%%%%%%%%%%%%%%%
\chapter{Accessing a web-service}
TODO

%%%%%%%%%%%%%%%%%%%%%%%%%%%%%%%%%%%
\chapter{Creating a web-service}
TODO

\section{Context}
If your REST service components require a context, e.g., flow context, then
you have to use the following approach for setting up the service
factory\footnote{\url{http://cxf.apache.org/docs/jaxrs-services-configuration.html}{}}:
\begin{verbatim}
JAXRSServerFactoryBean sf = new JAXRSServerFactoryBean();
CustomerService cs = new CustomerService();
sf.setServiceBeans(cs);
sf.setAddress("http://localhost:9080/");
sf.create();
\end{verbatim}
By instantiating the \textit{beans} yourself rather than through the factory,
you can provide them with context. You then use the \textit{setServiceBean(s)}
methods to set one or more beans.

%%%%%%%%%%%%%%%%%%%%%%%%%%%%%%%%%%%
\chapter{Flow}
This module contains generic actors in which you can simply plug your 
web-services that you have implemented. In the following a short overview.

The following standalones are available:
\begin{tight_itemize}
	\item \textit{RESTServer} -- runs a web-service. waiting for
	requests\footnote{adams-rest-server.flow}.
\end{tight_itemize}
The following sources are available:
\begin{tight_itemize}
	\item \textit{RESTSource} -- queries a web-service and forwards the received
	data\footnote{adams-rest-echo\_source.flow}.
\end{tight_itemize}
The following transformers are available:
\begin{tight_itemize}
	\item \textit{RESTTransformer} -- sends the data it receives to a web-service
	and forwards the data from the response in
	turn\footnote{adams-rest-echo\_transformer.flow}.
\end{tight_itemize}
The following sinks are available:
\begin{tight_itemize}
	\item \textit{RESTSink} -- simply sends data to a web-service\footnote{adams-rest-echo\_sink.flow}.
\end{tight_itemize}


%%%%%%%%%%%%%%%%%%%%%%%%%%%%%%%%%%%
% Copyright (c) 2009-2012 by the University of Waikato, Hamilton, NZ. 
% This work is made available under the terms of the 
% Creative Commons Attribution-ShareAlike 3.0 license, 
% http://creativecommons.org/licenses/by-sa/3.0/. 
%
% Version: $Revision: 3353 $

\begin{thebibliography}{999}
	% to make the bibliography appear in the TOC
	\addcontentsline{toc}{chapter}{Bibliography}

    % references
	\bibitem{adams}
		\textit{ADAMS} -- Advanced Data mining and Machine learning System \\
		\url{https://adams.cms.waikato.ac.nz/}{}

	\bibitem{ffmpeg}
		\textit{FFmpeg} -- a complete, cross-platform solution to record,
		convert and stream audio and video \\
		\url{http://ffmpeg.org/}{}

	\bibitem{xuggle}
		\textit{xuggle} -- a free open-source library for Java developers
		to uncompress, manipulate, and compress recorded or live video in real time \\
		\url{http://www.xuggle.com/}{}

	\bibitem{gzip}
		\textit{gzip} -- compression/decompression algorithm based on
		the DEFLATE algorithm, which is a combination of LZ77 and Huffman
		coding. \\
		\url{https://en.wikipedia.org/wiki/Gzip}{}

\end{thebibliography}


\end{document}
