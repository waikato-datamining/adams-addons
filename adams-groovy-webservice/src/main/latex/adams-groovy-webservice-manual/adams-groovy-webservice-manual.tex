% This work is made available under the terms of the
% Creative Commons Attribution-ShareAlike 4.0 license,
% http://creativecommons.org/licenses/by-sa/4.0/.

\documentclass[a4paper]{book}

\usepackage{wrapfig}
\usepackage{graphicx}
\usepackage{hyperref}
\usepackage{multirow}
\usepackage{scalefnt}
\usepackage{tikz}

% watermark -- for draft stage
%\usepackage[firstpage]{draftwatermark}
%\SetWatermarkLightness{0.9}
%\SetWatermarkScale{5}

\input{latex_extensions}

\title{
  \textbf{ADAMS} \\
  {\Large \textbf{A}dvanced \textbf{D}ata mining \textbf{A}nd \textbf{M}achine
  learning \textbf{S}ystem} \\
  {\Large Module: adams-groovy-webservice} \\
  \vspace{1cm}
  \includegraphics[width=2cm]{images/groovy-webservice-module.png} \\
}
\author{
  Peter Reutemann
}

\setcounter{secnumdepth}{3}
\setcounter{tocdepth}{3}

\begin{document}

\begin{titlepage}
\maketitle

\thispagestyle{empty}
\center
\begin{table}[b]
	\begin{tabular}{c l l}
		\parbox[c][2cm]{2cm}{\copyright 2016-2019} &
		\parbox[c][2cm]{5cm}{\includegraphics[width=5cm]{images/coat_of_arms.pdf}} \\
	\end{tabular}
	\includegraphics[width=12cm]{images/cc.png} \\
\end{table}

\end{titlepage}

\tableofcontents
%\listoffigures
%\listoftables

%%%%%%%%%%%%%%%%%%%%%%%%%%%%%%%%%%%
\chapter{Introduction}
The \textit{adams-groovy-webservice} module is a very light-weight module,
which allows you to query webservices using the Groovy\cite{groovy} scripting
language. It is merely aimed at providing client-side support,
not for implementing the server-side of webservices. If you want to also
implement the server-side, you should take a look at the \textit{adams-webservice}
module.

As underlying library it uses the \textit{groovy-wslite}\cite{groovy-wslite}
library.

%%%%%%%%%%%%%%%%%%%%%%%%%%%%%%%%%%%
\chapter{Flow}

In order to use webservices, you can simply use the existing Groovy actors
that are available from the \textit{adams-groovy} module. In these actors,
you need to import the groovy-wslite classes as follows:
\begin{verbatim}
import wslite.soap.*
\end{verbatim}
And in the \texttt{doExecute} method, you instantiate the client and query the
webservice.

Below is an example of querying a webservice for NZ public holidays
holidays\footnote{\url{http://kayaposoft.com/enrico/ws/}{}} in 2020:
{\scriptsize
\begin{verbatim}
def client = new SOAPClient('https://kayaposoft.com/enrico/ws/v2.0/index.php')
def response = client.send(SOAPAction:'https://kayaposoft.com/enrico/ws/v2.0/index.php') {
  body {
    getHolidaysForYear('xmlns':'http://www.kayaposoft.com/enrico/ws/v2.0/') {
      year(2020)     // year
      country("nz")  // New Zealand
      region("wko")  // Waikato
      holidayType("public_holiday")   // the type
    }
  }
}
\end{verbatim}
}
\noindent This example is also used in the example flow
\textit{adams-groovy-webservice\_nzholidays.flow} and the groovy script
\textit{adams-groovy-webservice\_nzholidays.groovy}.


%%%%%%%%%%%%%%%%%%%%%%%%%%%%%%%%%%%
% Copyright (c) 2009-2012 by the University of Waikato, Hamilton, NZ. 
% This work is made available under the terms of the 
% Creative Commons Attribution-ShareAlike 3.0 license, 
% http://creativecommons.org/licenses/by-sa/3.0/. 
%
% Version: $Revision: 3353 $

\begin{thebibliography}{999}
	% to make the bibliography appear in the TOC
	\addcontentsline{toc}{chapter}{Bibliography}

    % references
	\bibitem{adams}
		\textit{ADAMS} -- Advanced Data mining and Machine learning System \\
		\url{https://adams.cms.waikato.ac.nz/}{}

	\bibitem{ffmpeg}
		\textit{FFmpeg} -- a complete, cross-platform solution to record,
		convert and stream audio and video \\
		\url{http://ffmpeg.org/}{}

	\bibitem{xuggle}
		\textit{xuggle} -- a free open-source library for Java developers
		to uncompress, manipulate, and compress recorded or live video in real time \\
		\url{http://www.xuggle.com/}{}

	\bibitem{gzip}
		\textit{gzip} -- compression/decompression algorithm based on
		the DEFLATE algorithm, which is a combination of LZ77 and Huffman
		coding. \\
		\url{https://en.wikipedia.org/wiki/Gzip}{}

\end{thebibliography}


\end{document}
