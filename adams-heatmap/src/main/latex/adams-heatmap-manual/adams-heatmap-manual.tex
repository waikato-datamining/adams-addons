% Copyright (c) 2012 by the University of Waikato, Hamilton, NZ. 
% This work is made available under the terms of the 
% Creative Commons Attribution-ShareAlike 3.0 license, 
% http://creativecommons.org/licenses/by-sa/3.0/. 
%
% Version: $Revision$

\documentclass[a4paper]{book}

\usepackage{wrapfig}
\usepackage{graphicx}
\usepackage{hyperref}
\usepackage{multirow}
\usepackage{scalefnt}
\usepackage{tikz}

% watermark -- for draft stage
\usepackage[firstpage]{draftwatermark}
\SetWatermarkLightness{0.9}
\SetWatermarkScale{5}

\input{latex_extensions}

\title{
  \textbf{ADAMS} \\
  {\Large \textbf{A}dvanced \textbf{D}ata mining \textbf{A}nd \textbf{M}achine
  learning \textbf{S}ystem} \\
  {\Large Module: adams-heatmap} \\
  \vspace{1cm}
  \includegraphics[width=2cm]{images/heatmap-module.png} \\
}
\author{
  Peter Reutemann
}

\setcounter{secnumdepth}{3}
\setcounter{tocdepth}{3}

\begin{document}

\begin{titlepage}
\maketitle

\thispagestyle{empty}
\center
\begin{table}[b]
	\begin{tabular}{c l l}
		\parbox[c][2cm]{2cm}{\copyright 2011-2013} &
		\parbox[c][2cm]{5cm}{\includegraphics[width=5cm]{images/coat_of_arms.pdf}} \\
	\end{tabular}
	\includegraphics[width=12cm]{images/cc.png} \\
\end{table}

\end{titlepage}

\tableofcontents
\listoffigures
%\listoftables

%%%%%%%%%%%%%%%%%%%%%%%%%%%%%%%%%%%
\chapter{Introduction}
According to WikiPedia \cite{heatmap}, a ``heat map is a graphical representation of data 
where the individual values contained in a matrix are represented as 
colors.''

%%%%%%%%%%%%%%%%%%%%%%%%%%%%%%%%%%%
\chapter{Flow}
The following conversions are available:
\begin{tight_itemize}
  	\item \textit{BufferedImageToHeatmap} -- turns a BufferedImage into a heat
  	map, using the RGB values (but not alpha).
  	\item \textit{HeatmapToArray} -- generates a double array from a heat map.
  	\item \textit{HeatmapToBufferedImage} -- generates an image from heat map.
  	\item \textit{HeatmapToSpreadSheet} -- converts the heat map into a 
  	spreadsheet object.
  	\item \textit{SpreadSheetToHeatmap} -- creates a heat map from an all-numeric
  	spreadsheet.
\end{tight_itemize}
The following transformers are available:
\begin{tight_itemize}
  	\item \textit{HeatmapFileReader} -- reads a heat map from disk with a 
  	specified reader.
  	\item \textit{HeatmapFileWriter} -- writes a heat map back to disk with
  	a custom writer.
  	\item \textit{HeatmapFilter} -- transform a heat map using a filter.
  	\item \textit{HeatmapInstanceGenerator} -- turns a heat map into a WEKA
  	instance.
\end{tight_itemize}
The following sinks are available:
\begin{tight_itemize}
  	\item \textit{HeatmapDisplay} -- displays a heatmap.
\end{tight_itemize}

%%%%%%%%%%%%%%%%%%%%%%%%%%%%%%%%%%%
% Copyright (c) 2009-2012 by the University of Waikato, Hamilton, NZ. 
% This work is made available under the terms of the 
% Creative Commons Attribution-ShareAlike 3.0 license, 
% http://creativecommons.org/licenses/by-sa/3.0/. 
%
% Version: $Revision: 3353 $

\begin{thebibliography}{999}
	% to make the bibliography appear in the TOC
	\addcontentsline{toc}{chapter}{Bibliography}

    % references
	\bibitem{adams}
		\textit{ADAMS} -- Advanced Data mining and Machine learning System \\
		\url{https://adams.cms.waikato.ac.nz/}{}

	\bibitem{ffmpeg}
		\textit{FFmpeg} -- a complete, cross-platform solution to record,
		convert and stream audio and video \\
		\url{http://ffmpeg.org/}{}

	\bibitem{xuggle}
		\textit{xuggle} -- a free open-source library for Java developers
		to uncompress, manipulate, and compress recorded or live video in real time \\
		\url{http://www.xuggle.com/}{}

	\bibitem{gzip}
		\textit{gzip} -- compression/decompression algorithm based on
		the DEFLATE algorithm, which is a combination of LZ77 and Huffman
		coding. \\
		\url{https://en.wikipedia.org/wiki/Gzip}{}

\end{thebibliography}


\end{document}
