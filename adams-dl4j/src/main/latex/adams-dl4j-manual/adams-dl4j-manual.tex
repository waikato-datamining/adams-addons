% This work is made available under the terms of the
% Creative Commons Attribution-ShareAlike 4.0 license,
% http://creativecommons.org/licenses/by-sa/4.0/.
%
% Version: $Revision$

\documentclass[a4paper]{book}

\usepackage{wrapfig}
\usepackage{graphicx}
\usepackage{hyperref}
\usepackage{multirow}
\usepackage{scalefnt}
\usepackage{tikz}

% watermark -- for draft stage
\usepackage[firstpage]{draftwatermark}
\SetWatermarkLightness{0.9}
\SetWatermarkScale{5}

\input{latex_extensions}

\title{
  \textbf{ADAMS} \\
  {\Large \textbf{A}dvanced \textbf{D}ata mining \textbf{A}nd \textbf{M}achine
  learning \textbf{S}ystem} \\
  {\Large Module: adams-dl4j} \\
  \vspace{1cm}
  \includegraphics[width=2cm]{images/dl4j-module.png} \\
}
\author{
  Peter Reutemann
}

\setcounter{secnumdepth}{3}
\setcounter{tocdepth}{3}

\begin{document}

\begin{titlepage}
\maketitle

\thispagestyle{empty}
\center
\begin{table}[b]
	\begin{tabular}{c l l}
		\parbox[c][2cm]{2cm}{\copyright 2016} &
		\parbox[c][2cm]{5cm}{\includegraphics[width=5cm]{images/coat_of_arms.pdf}} \\
	\end{tabular}
	\includegraphics[width=12cm]{images/cc.png} \\
\end{table}

\end{titlepage}

\tableofcontents
%\listoffigures
%\listoftables

%%%%%%%%%%%%%%%%%%%%%%%%%%%%%%%%%%%
\chapter{Introduction}
The \textit{dl4j} module makes the deeplearning4j\cite{dl4j} library available
within ADAMS.

Due to the complex nature of configuring deep belief networks and such
multi-layered neural networks, there is no graphical support for configuring
networks as such. Instead, a so-called \textit{model configurator} is used
to return a configured model. This can be achieved in two ways:
\begin{tight_itemize}
  \item \textit{Custom Java code} -- you can subclass
  \textit{AbstractModelConfigurator} (located in package \textit{adams.ml.dl4j.model})
  and implement your own parameters that are necessary for tweaking your network. This is
  done by the \textit{SimpleMultiLayerNetwork} class in that package, which
  is just a configurable version of the network describe by deeplearning4j's
  tutorial on the iris dataset.
  \item \textit{Scripting} -- taking advantage of either Groovy or Jython,
  you can use the \textit{ModelWithScriptedConfiguration} class pointing
  to your script file that generates the actual network to be trained and used.
\end{tight_itemize}

%%%%%%%%%%%%%%%%%%%%%%%%%%%%%%%%%%%
\chapter{Flow}

The following sources are available:
\begin{tight_itemize}
  \item \textit{DL4JDatasetIterator} -- outputs datasets using the specified
  dataset iterator.
  \item \textit{DL4JModelConfigurator} -- defines a deeplearning model.
\end{tight_itemize}

The following transformers are available:
\begin{tight_itemize}
  \item \textit{DL4JRandomSplit} -- generates a random split on the incoming dataset.
  \item \textit{DL4JDatasetInfo} -- outputs information about a dataset.
  \item \textit{DL4JEvaluationSummary} -- generates a summary from an Evaluation
  object/container.
  \item \textit{DL4JEvaluationValues} -- retrieves statistics from an Evaluation
  object/container and places them in a spreadsheet.
  \item \textit{DL4JModelReader} -- reads a serialized model from disk.
  \item \textit{DL4JScoring} -- generates scores (= predictions) for a dataset
  using a built model.
  \item \textit{DL4JTestSetEvaluator} -- evaluates an incoming built model on a
  callable dataset.
  \item \textit{DL4JTrainModel} -- builds a model obtained from a callable model
  configurator on the incoming dataset.
  \item \textit{DL4JTrainTestSetEvaluator} -- trains and evaluates a model obtained
  from a callable model configurator on the incoming train/test set container.
\end{tight_itemize}

The following sinks are available:
\begin{tight_itemize}
  \item \textit{DL4JModelWriter} -- saves a model to disk using serialization.
\end{tight_itemize}

The following conversions are available:
\begin{tight_itemize}
  \item \textit{DL4JModelToJson} -- turns a model into its JSON\cite{json} representation.
  \item \textit{DL4JModelToYaml} -- turns a model into its YAML\cite{yaml} representation.
  \item \textit{NDArrayToSpreadSheet} -- turns a NDArray into a spreadsheet.
\end{tight_itemize}

%%%%%%%%%%%%%%%%%%%%%%%%%%%%%%%%%%%
% Copyright (c) 2009-2012 by the University of Waikato, Hamilton, NZ. 
% This work is made available under the terms of the 
% Creative Commons Attribution-ShareAlike 3.0 license, 
% http://creativecommons.org/licenses/by-sa/3.0/. 
%
% Version: $Revision: 3353 $

\begin{thebibliography}{999}
	% to make the bibliography appear in the TOC
	\addcontentsline{toc}{chapter}{Bibliography}

    % references
	\bibitem{adams}
		\textit{ADAMS} -- Advanced Data mining and Machine learning System \\
		\url{https://adams.cms.waikato.ac.nz/}{}

	\bibitem{ffmpeg}
		\textit{FFmpeg} -- a complete, cross-platform solution to record,
		convert and stream audio and video \\
		\url{http://ffmpeg.org/}{}

	\bibitem{xuggle}
		\textit{xuggle} -- a free open-source library for Java developers
		to uncompress, manipulate, and compress recorded or live video in real time \\
		\url{http://www.xuggle.com/}{}

	\bibitem{gzip}
		\textit{gzip} -- compression/decompression algorithm based on
		the DEFLATE algorithm, which is a combination of LZ77 and Huffman
		coding. \\
		\url{https://en.wikipedia.org/wiki/Gzip}{}

\end{thebibliography}


\end{document}
