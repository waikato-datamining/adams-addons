% This work is made available under the terms of the
% Creative Commons Attribution-ShareAlike 4.0 license,
% http://creativecommons.org/licenses/by-sa/4.0/.
%
% Version: $Revision: 12016 $

\documentclass[a4paper]{book}

\usepackage{wrapfig}
\usepackage{graphicx}
\usepackage{hyperref}
\usepackage{multirow}
\usepackage{scalefnt}
\usepackage{tikz}

% watermark -- for draft stage
%\usepackage[firstpage]{draftwatermark}
%\SetWatermarkLightness{0.9}
%\SetWatermarkScale{5}

\input{latex_extensions}

\title{
  \textbf{ADAMS} \\
  {\Large \textbf{A}dvanced \textbf{D}ata mining \textbf{A}nd \textbf{M}achine
  learning \textbf{S}ystem} \\
  {\Large Module: adams-nlp} \\
  \vspace{1cm}
  \includegraphics[width=2cm]{images/nlp-module.png} \\
}
\author{
  Peter Reutemann
}

\setcounter{secnumdepth}{3}
\setcounter{tocdepth}{3}

\begin{document}

\begin{titlepage}
\maketitle

\thispagestyle{empty}
\center
\begin{table}[b]
	\begin{tabular}{c l l}
		\parbox[c][2cm]{2cm}{\copyright 2013-2019} &
		\parbox[c][2cm]{5cm}{\includegraphics[width=5cm]{images/coat_of_arms.pdf}} \\
	\end{tabular}
	\includegraphics[width=12cm]{images/cc.png} \\
\end{table}

\end{titlepage}

\tableofcontents
\listoffigures
%\listoftables

%%%%%%%%%%%%%%%%%%%%%%%%%%%%%%%%%%%
\chapter{Introduction}
\textit{The Stanford Parser} is a GPL-licensed\footnote{version 2 or later \url{http://www.gnu.org/licenses/gpl-2.0.html}{}} 
library for natural language parsing.

\noindent For instance, parsing this sentence:
\begin{verbatim}
  The quick brown fox jumps over the lazy dog
\end{verbatim}
will result in a parse tree like this:
\begin{verbatim}
  (ROOT (NP (NP (DT The) (JJ quick) (JJ brown) (NN fox)) (NP (NP (NNS jumps)) 
    (PP (IN over) (NP (DT the) (JJ lazy) (NN dog))))))
\end{verbatim}
The same parse tree in graphical representation:
\begin{figure}[htb]
  \centering
  \includegraphics[width=4.0cm]{images/parse-tree.png}
  \caption{Graphical representation of Stanford parse tree.}
  \label{parse-tree}
\end{figure}

%%%%%%%%%%%%%%%%%%%%%%%%%%%%%%%%%%%
\chapter{Flow}
The following actors are available:
\begin{tight_itemize}
	\item \textit{CoNLLFileReader} -- transformer parsing text files in
	CoNLL format with one or more token sequences and turning them into
	spreadsheets.
	\item \textit{DocumentToSentences} -- transformer for splitting document
	strings into sentences using a specified
	splitter/tokenizer.\footnote{adams-nlp-split\_document\_and\_parse.flow}
	\item \textit{EditDistance} -- computes the edit distance between
	a supplied base string and the strings received.
	\item \textit{GenerateWordCloud} -- transformer for generating a word
	cloud image from word frequencies.
	\item \textit{StanfordGrammaticalStructure} -- transformer for generating a
	grammatical structure from a parse tree.\footnote{adams-nlp-parse.flow}
	\item \textit{StanfordLexicalizedParser} -- transformer for generating a 
	parse tree from a string.\footnote{adams-nlp-parse.flow}
	\item \textit{StanfordParseTreeDisplay} -- sink for displaying a parse 
	tree.\footnote{adams-nlp-parse.flow}
	\item \textit{Stemmer} -- transformer for performing stemming on words.
	\item \textit{Tokenize} -- transformer for splitting strings
	into words.
	\item \textit{TweeboParser} -- transformer for POS tagging tweets
	(requires the external TweeboParser executables).
	\item \textit{TweetNLPTagger} -- transformer for tokenizing/tagging
	tweets.
	\item \textit{WordFrequencyAnalyzer} -- transformer for generating word
	frequencies from text.
\end{tight_itemize}
The following conversions are available:
\begin{tight_itemize}
   	\item \textit{SpreadSheetToWordFrequencies} -- turns a spreadsheet
   	with two columns (word/frequency) into a word frequency array.
   	\item \textit{StanfordParseTreeToSpreadSheet} -- turns the leaves of a
   	parse tree into a spreadsheet.
	\item \textit{StanfordParseTreeToXML} -- turns a parse tree into an
	XML string.\footnote{adams-nlp-parse.flow}
   	\item \textit{WordFrequenciesToSpreadShet} -- turns a word frequency
   	array into a spreadsheet.
   	\item \textit{WordFrequencyToString} -- generates a string representation
   	of a single word frequency object.
\end{tight_itemize}
The following Weka tokenizers are available:
\begin{tight_itemize}
	\item \textit{TwitterNLPTokenizer} -- uses TwitterNLP's Twokenize.
\end{tight_itemize}
The following Weka filters are available:
\begin{tight_itemize}
	\item \textit{unsupervised.attribute.TwitterNLPPos} -- batch filter
	that adds frequencies of tag terms as determined by the TwitterNLP
	POS tagger.
\end{tight_itemize}

%%%%%%%%%%%%%%%%%%%%%%%%%%%%%%%%%%%
% Copyright (c) 2009-2012 by the University of Waikato, Hamilton, NZ. 
% This work is made available under the terms of the 
% Creative Commons Attribution-ShareAlike 3.0 license, 
% http://creativecommons.org/licenses/by-sa/3.0/. 
%
% Version: $Revision: 3353 $

\begin{thebibliography}{999}
	% to make the bibliography appear in the TOC
	\addcontentsline{toc}{chapter}{Bibliography}

    % references
	\bibitem{adams}
		\textit{ADAMS} -- Advanced Data mining and Machine learning System \\
		\url{https://adams.cms.waikato.ac.nz/}{}

	\bibitem{ffmpeg}
		\textit{FFmpeg} -- a complete, cross-platform solution to record,
		convert and stream audio and video \\
		\url{http://ffmpeg.org/}{}

	\bibitem{xuggle}
		\textit{xuggle} -- a free open-source library for Java developers
		to uncompress, manipulate, and compress recorded or live video in real time \\
		\url{http://www.xuggle.com/}{}

	\bibitem{gzip}
		\textit{gzip} -- compression/decompression algorithm based on
		the DEFLATE algorithm, which is a combination of LZ77 and Huffman
		coding. \\
		\url{https://en.wikipedia.org/wiki/Gzip}{}

\end{thebibliography}


\end{document}
