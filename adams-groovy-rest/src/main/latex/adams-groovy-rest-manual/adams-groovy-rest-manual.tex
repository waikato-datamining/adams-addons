% This work is made available under the terms of the
% Creative Commons Attribution-ShareAlike 4.0 license,
% http://creativecommons.org/licenses/by-sa/4.0/.

\documentclass[a4paper]{book}

\usepackage{wrapfig}
\usepackage{graphicx}
\usepackage{hyperref}
\usepackage{multirow}
\usepackage{scalefnt}
\usepackage{tikz}

% watermark -- for draft stage
%\usepackage[firstpage]{draftwatermark}
%\SetWatermarkLightness{0.9}
%\SetWatermarkScale{5}

\input{latex_extensions}

\title{
  \textbf{ADAMS} \\
  {\Large \textbf{A}dvanced \textbf{D}ata mining \textbf{A}nd \textbf{M}achine
  learning \textbf{S}ystem} \\
  {\Large Module: adams-groovy-rest} \\
  \vspace{1cm}
  \includegraphics[width=2cm]{images/groovy-rest-module.png} \\
}
\author{
  Peter Reutemann
}

\setcounter{secnumdepth}{3}
\setcounter{tocdepth}{3}

\begin{document}

\begin{titlepage}
\maketitle

\thispagestyle{empty}
\center
\begin{table}[b]
	\begin{tabular}{c l l}
		\parbox[c][2cm]{2cm}{\copyright 2019} &
		\parbox[c][2cm]{5cm}{\includegraphics[width=5cm]{images/coat_of_arms.pdf}} \\
	\end{tabular}
	\includegraphics[width=12cm]{images/cc.png} \\
\end{table}

\end{titlepage}

\tableofcontents
%\listoffigures
%\listoftables

%%%%%%%%%%%%%%%%%%%%%%%%%%%%%%%%%%%
\chapter{Introduction}
Developing REST webservices using JAX-RS\cite{jax-rs} is rather easy. However,
during development, services can go through many iterations. Having to deploy
a new build every single time can be time-consuming (and annoying).
In order to fill the gap, the \textit{adams-groovy-rest} module allows you
to write your REST plugins in Groovy\cite{groovy}, therefore avoiding the
recompilation of code, resulting in a faster turn-around time for your services.

%%%%%%%%%%%%%%%%%%%%%%%%%%%%%%%%%%%
\chapter{Flow}
Use the following \textit{RESTProvider} in your \textit{RESTServer} standalone:
\begin{verbatim}
adams.flow.rest.GroovyServer
\end{verbatim}
\noindent This provider allows you to point to an arbitrary number of Groovy
scripts that make up your REST service. It automatically sets the flow context
if the scripts should implement the \textit{adams.flow.core.FlowContextHandler}
interface. It also propagates its own logging level to the scripts, if they
should implement the \textit{adams.core.logging.LoggingLevelHandler} interface.

\newpage
\section{Writing a plugin}
Each of the Groovy scripts needs to contain a single class which either
implements the \textit{RESTPlugin} interface or is derived from the
\textit{AbstractRESTPlugin} or \textit{AbstractRESTPluginWithFlowContext}
super classes (the latter should be used if you require access to variables or
internal storage in the flow). The rest of the code uses the same JAX-RS annotations
as for writing a Java class.

Below is an example of a simple \textit{echo} client, which just sends
back the data it received as part of the URL of the
query\footnote{adams-groovy-rest\_echo.groovy}:
\begin{verbatim}
import adams.flow.rest.AbstractRESTPlugin
import javax.ws.rs.GET
import javax.ws.rs.Path
import javax.ws.rs.PathParam
import javax.ws.rs.Produces

class Echo extends AbstractRESTPlugin {

  @Override
  String globalInfo() {
    return "simple echo server"
  }

  @GET
  @Path("/echo/{input}")
  @Produces("text/plain")
  public String ping(@PathParam("input") String input) {
    getLogger().info("input: " + input)
    return input
  }
}
\end{verbatim}

%%%%%%%%%%%%%%%%%%%%%%%%%%%%%%%%%%%
% Copyright (c) 2009-2012 by the University of Waikato, Hamilton, NZ. 
% This work is made available under the terms of the 
% Creative Commons Attribution-ShareAlike 3.0 license, 
% http://creativecommons.org/licenses/by-sa/3.0/. 
%
% Version: $Revision: 3353 $

\begin{thebibliography}{999}
	% to make the bibliography appear in the TOC
	\addcontentsline{toc}{chapter}{Bibliography}

    % references
	\bibitem{adams}
		\textit{ADAMS} -- Advanced Data mining and Machine learning System \\
		\url{https://adams.cms.waikato.ac.nz/}{}

	\bibitem{ffmpeg}
		\textit{FFmpeg} -- a complete, cross-platform solution to record,
		convert and stream audio and video \\
		\url{http://ffmpeg.org/}{}

	\bibitem{xuggle}
		\textit{xuggle} -- a free open-source library for Java developers
		to uncompress, manipulate, and compress recorded or live video in real time \\
		\url{http://www.xuggle.com/}{}

	\bibitem{gzip}
		\textit{gzip} -- compression/decompression algorithm based on
		the DEFLATE algorithm, which is a combination of LZ77 and Huffman
		coding. \\
		\url{https://en.wikipedia.org/wiki/Gzip}{}

\end{thebibliography}


\end{document}
